\documentclass{ctexart}
\usepackage{listings}
\usepackage{fontspec}
\usepackage{geometry}

\geometry{margin=3cm}

\setmonofont{Menlo}[Scale=0.8]

\lstdefinestyle{mystyle}{
    basicstyle=\ttfamily,
    numbers=left,
    numbersep=5pt,
    showspaces=false,
    showstringspaces=false,
    showtabs=false,
    tabsize=2
}

\lstset{style=mystyle}

\author{}

\title{在Angular项目中实现Data Mocking的几种方式}

\begin{document}
    \maketitle

    \begin{abstract}
        在这篇文章中,我们会介绍几种基于Angular的一些特性实现的Data Mocking方法,
        通过这些方法,我们可以尽可能真实地模拟后端的开发环境,
        实现更加方便的前端开发、测试环境。具体地,我们会分别介绍:
        1) 基于 IoC 设计模式和 DI 技术的Data Mocking;
        2) 基于 HttpInterceptor 的Data Mocking;
        3) 基于工厂模式的配置切换方法。
        这些方法,解决了如何在 Angular 中灵活优雅地实现 Data Mocking 的问题,
        它的意义在于:为前端开发人员提供了一种可依赖的开发测试方式。
    \end{abstract}

    \section{引言}

    所谓Data Mocking,我们指的是,在前端以各种方式模拟后端,
    其中最主要的,就是模拟后端的响应(返回值),也能模拟一些简单的后端交互逻辑。
    通过这样的模拟(Simulation), 我们能提前知道我们写的程序到底是不是对的,
    我们能提前知道,我们有没有正确地处理后端返回的数据,以及\textbf{假如说}后端
    返回了正确的数据,我们能否做正确的处理。这些问题的答案很重要,因为当出现问题时,
    这些问题的答案能帮助我们快速定位问题。

    以上说的是简单的模拟,它得出的结果回答了「当后端按正常情况返回数据时,
    前端的表现是否也正常」的问题,Data Mocking也能做一些复杂的模拟,
    比如说模拟各种边界情况,对于一个按照约定本应是字符串的字段,
    我们可以模拟后端返回一个数字,一个数组或者一个布尔值,
    甚至模拟这种字段缺失的情况,以此来测试前端对于异常情况的处理如何,或者说前端
    的鲁棒性如何。压力测试(边界测试)实际上也属于这一类模拟,
    只不过后端返回的数据都是正常的,但是数据量远远地超出了正常的范围。

    往大了说,不管是前端也好,后端也好,在开发过程中,进行足够多的测试是有必要的,
    这就好比人们会通过打预防针接种疫苗,来获得对一些疾病的免疫能力,
    这其实就相当于软件开发中的自测,它不能够解决所有的问题,
    但是做了自测肯定比没做要好。

    \section{控制反转}

    控制反转(Inversion of Control, IoC)是一种开发实践,
    简单来说,在按照 IoC 思想设计的软件架构中,程序执行的控制权在框架,
    开发人员要做的,只是写好各个任务模块的代码,而框架则会在需要的时候去调用它们,
    这于传统的过程式编程的控制流是相反的——在传统的控制流中,程序员要自己决定
    先执行什么、后执行什么,而在按照 IoC 思想设计的的软件中,
    是框架去按需执行各个部分。

    \subsection{依赖注入}

    依赖注入(Dependency Injection)是一种设计模式,它实现了 IoC 开发实践。
    在 DI 设计模式中,有一个 IoC 容器负责初始化一个对象的各种依赖,开发人员
    只需把这种依赖声明出来。举例来说,当我们在 Angular 项目中需要发起 Http 请求时,
    有可能会这样写:

    \lstinputlisting{code/code-1.ts}

    这里,\verb`HttpClient` 是一个 Injection Token, 我们只是在构造器中告诉 IoC 容器:
    我们需要一个 HttpClient, 你去找吧,找来了顺便帮我实例化成 \verb`http` 并且注入到当前 \verb`HeroListComponent` 中供我使用。
    我们从前到后从来没有自己 \verb`new` 过这个 \verb`HttpClient`, 这都是 IoC 容器帮我们做的。

    如果觉得这个例子还是有些不理解的话,且看下一个例子:假设这回我们不打算在 \verb`Component` 里边做,
    而是希望将获取数据这个过程,交给 Service 来做,在此之前,我们先定好接口,
    也就是约定 Service 应该给我们获取到什么样的数据:

    \lstinputlisting{code/code-2.ts}

    好了,这样 Service 中实现的数据查询函数只要返给我们关于 \verb`IHeroQueryResult` 的 \verb`Observable` 就可以了,
    加载的过程是标准化的:

    \lstinputlisting{code/code-3.ts}

    我们首先对这样一个实现英雄查询功能的 Service 做个约定:

    \lstinputlisting{code/code-4.ts}

    然后我们做两个实现,一个是 Mock 的:

    \lstinputlisting{code/code-5.ts}
    
    一个是真正向后端发起 Http 请求的:

    \lstinputlisting{code/code-6.ts}

    在 Module 定义文件中,我们可以这样写:

    \lstinputlisting{code/code-7.ts}

    也可以这样写:

    \lstinputlisting{code/code-8.ts}

    这样对于在 \verb`HeroListComponent` 中出现的 \verb`HeroDataService` 这个 Injection Token, 
    R3 (Angular 负责实现依赖注入的模块) 会去 HeroModule 中查找到底是使用 \verb`HttpHeroDataService` 的实例来提供,
    还是使用 \verb`MockHeroDataService` 的实例来提供:

    \lstinputlisting{code/code-8-plus.ts}

    如果你觉得这个例子太过复杂,我们可以来看两个简单的例子,假设我们的产品有两个名称,一个内部名称,和一个外部名称,
    那么我们可以这样先定义一个 InjectionToken,以及写好两个产品名称:

    \lstinputlisting{code/code-9.ts}

    我们会在 Layout 中依赖它:

    \lstinputlisting{code/code-10.ts}

    当然,为了让 R3 能找到 \verb`LOGO` 这个 Injection Token 的「提供者」,还需要在 Module 中显式声明:

    \lstinputlisting{code/code-11.ts}

    这样,\verb`LayoutComponent` 依赖的那个 \verb`LOGO` 就会被 R3 用 \verb`INTERNAL_LOGO` 注入。

    \section{拦截器}

    Angular 的 http 模块提供了 \verb`HTTP_INTERCEPTORS` 这个 Injection Token, 它被定义为:

    \lstinputlisting[numbers=none]{code/code-12.ts}

    我们可以自己实现想要的 \verb`HttpInterceptor` 提供给 \verb`HTTP_INTERCEPTORS` 来实现 Http 请求拦截功能,这样做,
    可以在请求发出之前修改请求参数,也可以在请求返回以后修改返回结果。

    现在让我们来实现一个简单的 \verb`HttpInterceptor`, 它拦截符合条件的请求,并且返回 404:

    \lstinputlisting{code/code-13.ts}

    通过在 Hero 模块中以这种方式提供 \verb`HttpInterceptor`, 我们可以拦截发往 \verb`/api/v1/heroes` 的请求,并且无条件返回 404:

    \lstinputlisting{code/code-14.ts}

    注意到 \verb`multi: true`, 这是因为 \verb`HTTP_INTERCEPTORS` 是一个 \verb`HttpInterceptor[]`, 可以以这种方式被多次提供,也就是说,我们可以实现多个 \verb`HttpInterceptor`, 并且都提供给 \verb`HTTP_INTERCEPTORS`, 使得每个 \verb`HttpInterceptor` 都生效。

    \section{工厂模式}

    现在是,我们在线上和开发测试中使用两套后端,我们希望能实现自动切换。具体地,我们在一个配置文件中定义好后端地址:

    \lstinputlisting{code/code-15.ts}

    假设已经有了一个服务,它能够判断程序当前是运行在开发环境还是生产环境:

    \lstinputlisting{code/code-16.ts}

    那么我们就可以让程序自动地决定应该使用哪个后端:

    \lstinputlisting{code/code-17.ts}

    当有组件或者服务要用到 \verb`HTTP_BACKEND` 这个 Injection Token 的时候,R3 就会用工厂生产一个 \verb`HttpBackendConfig` 来提供,而这个工厂的原材料则是一个 \verb`EnvProbService`.


\end{document}
